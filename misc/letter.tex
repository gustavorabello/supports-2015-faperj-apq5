\typeout{ ====================================================================}
\typeout{ this is file letter.tex, created at 25-Mar-2015                     }
\typeout{ maintained by Gustavo Rabello dos Anjos                             }
\typeout{ e-mail: gustavo.rabello@gmail.com                                   }
\typeout{ ====================================================================}

\documentclass[12pt,a4paper]{article}
\usepackage[utf8]{inputenc}
\usepackage[portuguese]{babel}
\usepackage[top=3cm,bottom=3cm,left=3.7cm,right=3.7cm]{geometry}
\usepackage{graphicx}    % pacote para inclusao de figuras,
\usepackage{subfig}
\pagestyle{plain}


\begin{document}

\begin{tabular}{ccc}
 \begin{minipage}[c]{6cm}
  \begin{flushleft}
   \includegraphics[scale=0.055]{uerj-bw.png}\\
   \includegraphics[scale=1.0]{gesar.png}
  \end{flushleft}
 \end{minipage}
&
&
	\begin{minipage}{6.5cm}
		\begin{flushleft}
			\small
			\noindent GESAR/PPG-EM/UERJ \\
			Rua Fonseca Teles, 121 \\ 
			S\~ao Crist\'ov\~ao \\ 
			Rio de Janeiro -- Brazil\\
			tel.: +55 21 2332-4733\\
			{\tt gustavo.anjos@uerj.br}\\
			{\tt http://www.gesar.uerj.br}\\
			{\tt http://www.ppg-em.eng.uerj.br}\\
			\vspace*{2mm}
			25 março 2015
		\end{flushleft}
	\end{minipage}
\end{tabular}

\vspace*{1.5cm}
\noindent \textbf{Para}: Fundação de Amparo à Pesquisa do Estado do Rio
de Janeiro.\\
\noindent \textbf{Assunto}: Justificativa do Pedido de Apoio - AP5.\\

\noindent Prezado Sr/Sra.:\\

O Congresso InterPACK (International Technical Conference and Exhibition
on Packaging and Integration of Electronic and Photonic Microsystems) ao
qual estou solicitando apoio, é um congresso internacional para troca de
conhecimento em pesquisa, desenvolvimento, fabricação e aplicações
industriais de ponta em acondicionamento de dispositivos eletrônicos de
última geração. Esta é a conferência mais importante da Divisão de
Acondicionamento Eletrônico (EPPD) dos EUA, realizada desde 1992 pela
Sociedade Americana de Engenharia Mecânica (ASME). Esta conferência
atrai pesquisadores de diversas partes do mundo na área acadêmica,
industrial e governamental, tornando assim uma excelente oportunidade
para disseminação e colaboração entre os participantes, além de
apresentar caráter multidisciplinar, aliando diretamente os ramos da
engenharia mecânica e térmica à eletrônica. Os anais da conferência
serão publicados pela ASME no formato de trabalhos completos.

Minha trajetória acadêmica foi inspirada, em grande parte, pela
elaboração de modelos numéricos capazes de compreender em detalhes a
dissipação de calor em componentes eletrônicos submetidos a altas taxas
de geração de energia, tema este estudado em minha tese de doutorado e
realizado no renomado Laboratório de Transferência de Calor e Massa
(LTCM) da École Polytecnique Fédéral de Lausanne (EPFL), na Suíça. Desde
então, mantenho esta pesquisa ativa e uma estreita colaboração com o
Prof. John Thome, responsável pelo laboratório. Esta colaboração já
resultou em diversos artigos, incluindo este que foi submetido e aceito
pelo congresso InterPACK, como também a coorientação de uma tese de
doutorado que está em fase inicial na Suíça. Vale também destacar que
outros trabalhos frutos desta colaboração estreita estão em fase de
produção e serão submetidos brevemente à avaliação de revistas
internacionais especializadas.

Acredito fortemente que este congresso trará grandes benefícios
acadêmicos e maiores colaborações internacionais, impactando diretamente
no Programa de Pós-Graduação em Engenharia Mecânica (PPG-EM) da
Universidade do Estado do Rio de Janeiro (UERJ), programa este que faço
parte como professor permanente desde o início de meu vínculo
profissional com a universidade. Acredito também que a exposição deste
trabalho técnico resultará em grande visibilidade internacional de nossa
pesquisa científica, consolidando o programa de pós-graduação e a
universidade como centro de excelência em pesquisa de ponta e moderna.
Este trabalho é vinculado à pesquisa em nível de graduação através de
projeto de iniciação científica, sob minha orientação, do aluno Paulo
Roberto B. L. Filho.

\vspace{2cm}

\noindent Mantenho-me à disposição para eventuais esclarecimentos,

\vspace{2cm}

\begin{center}
Prof. Gustavo Anjos
\end{center}

\end{document}


\typeout{ ****************** End of file letter.tex ****************** }

